% This document is part of the AsteroseismicLikelihood project
% Copyright 2020 the authors.

% TODO:
% - foo

% Notes:
% - foo
% - bar

\PassOptionsToPackage{usenames,dvipsnames}{xcolor}
\documentclass[modern]{aastex63}
% \documentclass[twocolumn]{aastex63}

% Load common packages
\usepackage{microtype}  % ALWAYS!
\usepackage{amsmath}
\usepackage{amsfonts}
\usepackage{amssymb}
\usepackage{mathrsfs}
\usepackage{booktabs}
\usepackage{graphicx}

\usepackage{enumitem}
\setlist[description]{style=unboxed}

% Hogg's issues need to be addressed.
\renewcommand{\twocolumngrid}{\onecolumngrid} % guess what this does HAHAHA!
\setlength{\parindent}{1.1\baselineskip}
\addtolength{\topmargin}{-0.2in}
\addtolength{\textheight}{0.4in}
\sloppy\sloppypar\raggedbottom\frenchspacing

\shorttitle{asteroseismic likelihood function}
\shortauthors{bonaca and hogg}

\begin{document}

\title{A likelihood function for asteroseismic parameter measurement}

\author{Ana Bonaca}
\affiliation{Harvard}

\author[0000-0003-2866-9403]{David~W.~Hogg}
\affiliation{Center for Cosmology and Particle Physics,
             Department of Physics,
             New York University, 726 Broadway,
             New York, NY 10003, USA}
\affiliation{Max-Planck-Institut f\"ur Astronomie,
             K\"onigstuhl 17, D-69117 Heidelberg, Germany}
\affiliation{Center for Computational Astrophysics, Flatiron Institute,
             Simons Foundation, 162 Fifth Avenue, New York, NY 10010, USA}

\begin{abstract}\noindent
% Context
Asteroseismology generally proceeds by taking a fourier transform of a
well-sampled light curve and identifying normal modes, which are then organized
into the asteroseismic parameters nu-max, delta-nu, and others.
% Aims
Here we propose to replace this methodology with parameter inference with a
likelihood function.
An inference-based method would permit asteroseismic studies when light curves
are irregularly sampled and heterogeneous in quality.
% Methods
We create a likelihood function parameterized by a central frequency and a
frequency difference, in which the light curve is generated by $K$ coherent
frequencies determined by those parameters.
The delta-nu parameter is found by maximizing this likelihood.
Uncertainties are found by profiling.
% Results
The optimization is extremely difficult. But when it works, it's great!
Etc.
% Conclusions
\end{abstract}

% \keywords{}

\section*{~}\clearpage
\section{Introduction} \label{sec:intro}

foo and bar.

\section{Method}

The LF is...

The profile LF is...

\acknowledgments

It is a pleasure to thank
  Stephen Feeney (UCL)
  and
  Dan Foreman-Mackey (Flatiron)
for help with all these concepts.

This work has made use of data from the European Space Agency (ESA) mission
{\it Gaia} (\url{https://www.cosmos.esa.int/gaia}), processed by the {\it Gaia}
Data Processing and Analysis Consortium (DPAC,
\url{https://www.cosmos.esa.int/web/gaia/dpac/consortium}). Funding for the DPAC
has been provided by national institutions, in particular the institutions
participating in the {\it Gaia} Multilateral Agreement.

\software{
    % Astropy \citep{astropy, astropy:2018},
    % exoplanet \citep{exoplanet:exoplanet},
    % gala \citep{gala},
    % IPython \citep{ipython},
    % numpy \citep{numpy},
    % pymc3 \citep{Salvatier2016},
    % schwimmbad \citep{schwimmbad:2017},
    % scipy \citep{scipy},
    % theano \citep{theano},
    % thejoker \citep{thejoker, Price-Whelan:2019a}
}


% \appendix

% \section{Update to the marginal likelihood expression for \thejoker}
% \label{app:marginal-likelihood}


\bibliographystyle{aasjournal}
\bibliography{comoving-dr2}

\end{document}
